% https://www.monash.edu/graduate-research/examination/publication
% https://www.monash.edu/rlo/graduate-research-writing/write-the-thesis
% https://www.monash.edu/rlo/graduate-research-writing/write-the-thesis/writing-the-thesis-chapters/structuring-a-long-text
\abstract{
\addtocontents{toc}{}  % Add a gap in the Contents, for aesthetics

Zircaloy-4 and other hexagonal phase alloys develop strong and unavoidable crystallographic textures during conventional processing.
The effects of this texture are difficult to decouple from the fundamental underlying deformation mechanisms.
The texture results in highly anisotropic response, where loading along different directions results in vastly different yield behaviours.
Powder metallurgy and hot isostatic pressing (HIP) were used to produce hexagonal metals that have a fully random crystallographic texture.
The HIP process requires a long dwell at high temperature below the beta-transus, which leads to abnormal grain growth in some circumstances, however a successful HIP process was determined.
Calculated Kearns factors and the disorientation distribution confirm that a random texture was obtained.
Statistical geometric alignment between neighbour grains revealed a measureable difference between the strongly textured material and the HIPed material.
The presence of strong texture results in a statistical peak in well-aligned grains, which was not present for the random texture material, indicating a much higher average constraint by neighbouring grains.
Proposed future work for this project includes fitting crystal plasticity models




}