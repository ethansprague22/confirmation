\chapter{Research aims}

{\LARGE\textbf{Aims}}
\begin{enumerate}
    \item Create texture-free HCP alloys (Ti/Zr) using HIP
    \item Inform crystal plasticity model with mechanical testing results
    \item Gain deeper understanding of fundamental deformation behaviour, especially polycrystal neighbourhood effects using in-situ methods
    \item Use insights to develop novel behaviour law for crystal plasticity models
\end{enumerate}

\vspace{\baselineskip}
{\LARGE\textbf{Objectives}}

\textbf{HIP process:}
\begin{itemize}
    \item Determine working HIP parameters for producing desired material
    \item Consider how temperature and pressure profiles, dwell time, powder stock, can geometry, and packing density affect the final microstructure (porosity, grain size distribution)
\end{itemize}

\textbf{Perform comprehensive mechanical testing:}
\begin{itemize}
    \item Generate accurate and reliable experimental data from standard mechanical tests (tensile tests, cyclic loading, hot compression)
\end{itemize}

\pagebreak
\textbf{Inform CP-FFT model:}
\begin{itemize}
    \item Create virtual representative volume element using Dream3D, use MFront to define material behaviour law, use AMITEX to solve constitutive equations in FFT regime
    \item Perform parameter fitting using mechanical test data
    \item Evaluate model generality with respect to texture by comparing results for HIPed and conventionally processed material
\end{itemize}

\textbf{In-situ testing}
\begin{itemize}
    \item Use HR-DIC to track plastic strain during loading
    \item Employ Relative Displacement Ratio (RDR) analysis to identify active slip systems
    \item Correlate deformation to local grain neighbourhood
\end{itemize}

\textbf{Development of novel crystal plasticity behaviour law}
\begin{itemize}
    \item Incorporate grain boundary effects such as geometric compatibility
    \item Utilise composite voxels to accurately describe grain boundary geometry and incorporate complex interfacial effects
\end{itemize}