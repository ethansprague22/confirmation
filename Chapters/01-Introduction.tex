\chapter{Introduction}

The cladding for the nuclear fuel of pressurised water reactors is a critical component for which reliable integrity evaluations must be made over the entire fuel cycle, from service to waste storage.
Zirconium alloys are selected for the cladding material because of their low neutron absorption, good corrosion resistance and suitable mechanical properties.
It is crucial that the mechanical behaviour of these alloys is well understood, and accurate models are necessary for safety assessments.
Increasingly, advanced safety systems use a 'digital twin' model of the entire reactor which must accurately predict the mechanical response of the cladding material during simulated emergencies.

Zirconium alloys used for cladding have a hexagonal close packed (HCP) crystal structure.
These alloys always develop a strong crystallographic texture during conventional processing such as rolling and pilgering.
This crystallographic texture has a significant impact on the mechanical behaviour, resulting in a large anisotropy and directional dependence.
Furthermore, the texture impacts other properties such as the formation sites for hydrides, which preferentially grow along specific crystallographic planes.
Finally, polycrystalline deformation requires each grain to accommodate the deformation of its neighbour grains, even when there is a mismatch in the single crystal properties due differences in orientation.
These neighbour interactions must be taken into account during both in-service and storage conditions, where thermal cycling and variations in pressure can lead to heterogeneous deformation.

The seemingly unavoidable crystallographic texture which emerges from processing dominates samples that are used to study the behaviour of these alloys.
Since texture has such a large effect, it is possible that underlying behaviours are being obscured by texture effects.
Material modelling should be able to accurately predict the mechanical performance of these alloys for any given texture, however, these models have only ever been extensively tested against data from strongly textured material.

This project uses powder metallurgy techniques, namely hot isostatic pressing (HIP) to create HCP material with a truly random texture by eliminating all sources of directional stress.
Studying this material alongside its conventionally processed equivalent enables a far more robust approach to optimising existing material models, with the additional constraint of a randomly textured material.

This work is directed at understanding zircaloy-4, but the results will have broader impact more generally on similar hexagonal metals such as $\alpha$-titanium alloys.
The new in-situ testing capabilities at Monash Centre for Electron Microscopy provide the opportunity to understand the complex and asymmetric deformation within these hexagonal polycrystalline materials, allowing for a deeper understanding on fundamental deformation of an entire class of alloys.
Finally, this project opens the pathway for future study on other texture-related phenomenon such as the effect of hydrides in texture-free material.