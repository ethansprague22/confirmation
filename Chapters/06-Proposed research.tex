\chapter{Proposed research}

\section{Mechanical testing of HIPed material}
The most immediate next step in the project is the mechanical testing of the HIPed material.
Simple tensile tests using the same parameters as for the rolled plate samples, and using the same sample geometry.
In addition to simple uniaxial tensile testing, cycle loading tests alongside the rolled samples would allow for a comparison of the kinematic hardening behaviour between the two samples.
Post-mortem analysis of the microstructure after interrupted tensile tests would also provide insight into the role that deformation twinning plays during plasticity.

\section{HR-DIC in-situ testing}
Use of the new capability at Monash Centre for Electron Microscopy (MCEM) for in-situ tensile testing would allow for high resolution digital image correlation to be used to map the strain localisation throughout the microstructure as a function of total strain.
This technique, when paired with Relative Displacement Ratio (RDR) analysis would allow for identification of active slip modes and in-situ EBSD correlation would track the twinning behaviour as well as localised crystal rotation.

\section{Rolling HIPed material}
Direct comparison of the HIPed material to the purchased rolled plate is difficult as the exact alloy composition is not consistent, particularly the oxygen content, which has a significant effect on the mechanical properties.
To truly isolate crystallographic texture as the only independent variable, the HIPed material will be rolled to produce intermediate textures.
This will involve rolling to multiple different thickness reductions, and then calibrating a heat treatment regime for each of them to produce a recrystallised microstructure with constant grain size across all samples.
Mechanical testing and analysis of these samples would then truly highlight the transition between completely random texture to the presence of a strong crystallographic texture which is more akin to what is typically found in service.

\section{Crystal plasticity modelling}
The next stages in crystal plasticity modelling will be calibrating a twinning model to reproduce the anisotropic response from the rolled plate samples.
Once tensile data for the HIPed material is obtained, it can be incorporated into the fitting regime as an additional constraint.

Beyond this, novel constitutive behaviuour laws will be developed to incorporate a description of the geometric compatibility effect, especially in relation to long range networks of strain.
To account for complex interfacial interactions, composite voxels will be incorporated using AMITEX.

\section{Irradiation}
Irradiation is known to significantly enhance strain localisation in Zircaloy-4.
By irradiating random texture HCP, the neighbourhood dependence of strain localisation could be studied.

\section{X-ray / neutron diffraction in-situ loading}
